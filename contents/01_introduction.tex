
\todo[inline]{shorten / refine}

In recent years climate change has become one of the most important issues for humanity. 
To target climate change effectively, strategies need to be implemented in order to reduce carbon emissions and transition towards sustainable energy sources. 

Effective climate governance requires a polycentric approach, by increasing the role of non-state actor and employing diverse actions by multiple actors at various levels and industries \cite{Huitema2010GoverningCC}. Furthermore, local actions become important in climate governance, as small-scale initiatives can often be more effective and adaptable to specific circumstances. 
However, existing implementations of local policies and programs intended to govern  this global common are inefficient and highlight institutional weaknesses \cite{brazilCommons}. Not only is it challenging to implement effective monitoring of the transition to renewable energy sources but also the potential to infringe upon personal freedoms and privacy is a concern \cite{molina2010private}. Yet, this data collection is necesarry in order to applly proper incentives and punishment and  to assess the impact and effectiveness of climate policies and environmental initiatives. \info[inline]{add resources that underline the importance of measuring data in order to apply proper punishments and rewards} 

In this regard, building systems that empower local communitites and incentivise them to support the achievment of societal objectives enables effective polycentric climate governance. While each household’s energy consumption might seem minor in isolation, collectively, they form a substantial part of the overall energy footprint. by implementing compliance verification systems that allows for policy verification in local energy markets, the transisition towards renewables energy sources of communities can be tackeld more effectively.  This provides a tangible way to assess the impact and effectiveness of climate policies and environmental initiatives. Furthermore it enables the tracking of progress and provision of incentives, while also holding households accountable for their energy consumption and environmental impact.

This issue can be further complicated by the global increase in state monitoring and data collection in recent years, as Eck et al. \cite{surveill} note. They caution that such practices could pose threats to privacy rights and democratic norms, which emphasises the need for systems that can fulfil societal goals without compromising these fundamental principles.


Therefore, building systems that achieve a balance between compliance and privacy might be a key factor in gaining public trust and support for climate action policies. Furthermore, such systems can reduce state intervention and monitoring, while simultaneously empowering communities by encouraging shared responsibilities and collective-driven initiatives. 

This paper introduces a system that allows effective and verifiable compliance checks at a collective level, while still protecting individual privacy, as can be seen in Figure 1.1. In order to accomplish this, the nesting of Zero-Knowledge Proofs (ZKP) will be applied

We make the following three individual contributions:
\begin{itemize}
    \item
    \item
    \item
\end{itemize}

\todo[inline]{Paper outline here}
